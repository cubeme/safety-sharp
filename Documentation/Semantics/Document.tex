\documentclass{article}
\usepackage[utf8]{inputenc}
\usepackage{lmodern}
\usepackage[T1,hyphens]{url}
\usepackage[pdftex]{hyperref}
\usepackage{mathrsfs}
\usepackage{amsfonts}
\usepackage{stmaryrd}
\usepackage[fleqn,leqno]{amsmath}
\usepackage{amssymb}
\usepackage{mathtools}

% Names
\newcommand{\CSharp}{{\settoheight{\dimen0}{C}C\kern-.05em \resizebox{!}{\dimen0}{\raisebox{\depth}{\#}}}\xspace}
\newcommand{\SSharp}{{\settoheight{\dimen0}{S}S\kern-.05em \resizebox{!}{\dimen0}{\raisebox{\depth}{\#}}}\xspace}
\newcommand{\SafetySharp}{Safety\-Sharp\xspace}
\newcommand{\DotNet}{\textsc{.Net}\xspace}
\newcommand{\Fil}{\textsc{Fil}\xspace}
\newcommand{\Fm}{\textsc{Fm}\xspace}

% Domains
\newcommand{\Bool}{\mathbb{B}}
\newcommand{\Integer}{\mathbb{Z}}
\newcommand{\Var}{\mathit{Var}}
\newcommand{\Uop}{\mathit{UnaryOp}}
\newcommand{\Bop}{\mathit{BinaryOp}}
\newcommand{\Expr}{\mathit{Expr}}
\newcommand{\StmFil}{\mathit{Stm}_\Fil}
\newcommand{\StmFm}{\mathit{Stm}_\Fm}
\newcommand{\Prog}{\mathit{Prog}}
\newcommand{\GuardedCmd}{\mathit{GrdCmd}}
\newcommand{\VarEnvFil}{\Sigma_\Fil}
\newcommand{\VarEnvFm}{\Sigma_\Fm}
\newcommand{\FieldEnv}{\Phi}
\newcommand{\FieldEnvCurried}{\Phi\urcorner}
\newcommand{\Val}{\mathit{Value}}
\newcommand{\stmArrowFil}{\rightarrow_{\Fil}}
\newcommand{\stmArrowFm}{\xrightarrow{\compDecl}_\Fm}
\newcommand{\Type}{\mathit{Type}}
\newcommand{\Field}{\mathit{Field}}
\newcommand{\Local}{\mathit{Local}}
\newcommand{\ReqPort}{\mathit{ReqPort}}
\newcommand{\ProvPort}{\mathit{ProvPort}}
\newcommand{\Comp}{\mathit{Comp}}
\newcommand{\BindingDecl}{\mathit{BndDecl}}
\newcommand{\Procedure}{\mathit{Proc}}
\newcommand{\VarDecl}{\mathit{VarDecl}}
\newcommand{\FieldDecl}{\mathit{FieldDecl}}
\newcommand{\ReqPortDecl}{\mathit{ReqPortDecl}}
\newcommand{\ProvPortDecl}{\mathit{ProvPortDecl}}
\newcommand{\CompDecl}{\mathit{CompDecl}}
\newcommand{\BindingSrc}{\mathit{BndSrc}}
\newcommand{\BindingTarget}{\mathit{BndTarget}}
\newcommand{\BindingType}{\mathit{BndType}}

% Meta variables
\newcommand{\bool}{b}
\newcommand{\integer}{z}
\newcommand{\var}{v}
\newcommand{\uop}{\mathit{uop}}
\newcommand{\bop}{\mathit{bop}}
\newcommand{\expr}{e}
\newcommand{\stmFil}{S}
\newcommand{\stmFm}{S}
\newcommand{\prog}{P}
\newcommand{\guardedCmd}{G}
\newcommand{\varEnvFil}{\sigma}
\newcommand{\varEnvFm}{\varsigma}
\newcommand{\varEnvFmEmpty}{\varEnvFm_{{}_\failure}}
\newcommand{\fieldEnv}{\varphi}
\newcommand{\fieldEnvOld}{\varphi_p}
\newcommand{\fieldEnvCurried}{\phi}
\newcommand{\fieldEnvEmpty}{\fieldEnv_{{}_\failure}}
\newcommand{\val}{\nu}
\newcommand{\emptyList}{\varepsilon}
\newcommand{\type}{\tau}
\newcommand{\field}{f}
\newcommand{\local}{l}
\newcommand{\reqPort}{r}
\newcommand{\provPort}{p}
\newcommand{\comp}{c}
\newcommand{\bindingDecl}{\mathit{B}}
\newcommand{\procedure}{\mathit{proc}}
\newcommand{\varDecl}{V}
\newcommand{\fieldDecl}{F}
\newcommand{\reqPortDecl}{R}
\newcommand{\provPortDecl}{P}
\newcommand{\compDecl}{C}
\newcommand{\bindingSrc}{\mathit{bs}}
\newcommand{\bindingTarget}{\mathit{bt}}
\newcommand{\bindingType}{\mathtt{\rightarrowtail}}

%Expressions
\newcommand{\unaryExpr}[2]{#1 #2}
\newcommand{\binaryExpr}[3]{#2 #1 #3}
\newcommand{\prev}[1]{\mathtt{prev} \,#1}
\newcommand{\true}{\mathit{tt}}
\newcommand{\false}{\mathit{ff}}

% Statements
\newcommand{\skipStm}{\mathtt{skip}}
\newcommand{\assignVarStm}[2]{#1\; \mathtt{:=}\; #2}
\newcommand{\invokePort}[3]{#1 (#2; #3)}
\newcommand{\invokeBeh}[1]{#1 ()}
\newcommand{\seqStm}[2]{#1 \; \mathtt{;} \; #2}
\newcommand{\guardedClause}[2]{#1 \; \mathtt{\twoheadrightarrow} \; #2}
\newcommand{\guardedStm}[1]{(#1)^*}

% Formal model
\newcommand{\varDeclSyn}[2]{#1 \; \mathtt{:} \; #2}
\newcommand{\fieldDeclSyn}[3]{#1 \; \mathtt{:} \; #2 \; \mathtt{=} \; #3}
\newcommand{\reqPortDeclSyn}[3]{#1 \mathtt{(} #2 \; \mathtt{;} \; #3 \mathtt{)}}
\newcommand{\provPortDeclSyn}[5]{#1 \mathtt{(} #2 \; \mathtt{;} \; #3 \mathtt{) \; \{} \; #4 \; #5 \; \mathtt{\}}}
\newcommand{\bindingDeclSyn}[3]{#1 \; #3 \; #2}
\newcommand{\instantBinding}{\mathtt{\bindingType_i}}
\newcommand{\delayedBinding}{\mathtt{\bindingType_d}}
\newcommand{\qualifiedName}[2]{#1\mathtt{.}#2}
\newcommand{\compDeclSyn}[8]{\texttt{component} \; #1 \; \mathtt{\{} \; #2 \; #3 \; #4 \; #5 \; #6 \; \mathtt{\{} \; #7 \; #8 \;
\mathtt{\}} \; \mathtt{\}}}

% Lifting
\newcommand{\failure}{\bot}
\newcommand{\missing}{\varepsilon}
\newcommand{\liftMissing}[1]{#1_\missing}
\newcommand{\liftFailure}[1]{#1_\failure}
\newcommand{\liftMissingFailure}[1]{#1_{\missing, \failure}}
\newcommand{\configurationFil}[1]{\langle #1 \rangle}
\newcommand{\configurationFm}[1]{\langle #1 \rangle}
\newcommand{\toList}[1]{#1^*}
\newcommand{\ToList}[1]{#1^*}
\newcommand{\toListNE}[1]{#1^+}
\newcommand{\ToListNE}[1]{#1^+}
\newcommand{\toSet}[1]{\hat{#1}}
\newcommand{\ToSet}[1]{\wp #1}

% Functions
\newcommand{\asSet}{\mathit{set}}
\newcommand{\ExprSem}[1]{\mathcal{E} \llbracket #1 \rrbracket}
\newcommand{\StmSem}[1]{\mathcal{S} \llbracket #1 \rrbracket}
\newcommand{\ProgSem}[1]{\mathcal{P} \llbracket #1 \rrbracket}
\newcommand{\cons}{\circ}
\newcommand{\unionCons}{\bullet}
\newcommand{\varof}{\mathit{var}}
\newcommand{\fieldof}{\mathit{field}}
\newcommand{\nameof}{\mathit{name}}
\newcommand{\typeof}{\mathit{type}}
\newcommand{\signature}{\mathit{sig}}
\newcommand{\inputs}{\mathit{in}}
\newcommand{\outputs}{\mathit{out}}
\newcommand{\locals}{\mathit{loc}}
\newcommand{\stmof}{\mathit{stm}}
\newcommand{\req}{\mathit{req}}
\newcommand{\prov}{\mathit{prov}}
\newcommand{\delayed}{\mathit{delayed}}
\newcommand{\subsof}{\mathit{subs}}
\newcommand{\fields}{\mathit{fields}}
\newcommand{\compof}{\mathit{comp}}
\newcommand{\reqsof}{\mathit{reqs}}
\newcommand{\provsof}{\mathit{provs}}
\newcommand{\init}{\mathit{init}}
\newcommand{\defaultof}{\mathit{default}}
\newcommand{\boundof}{\mathit{bound}}

% Values
\newcommand{\mkTuple}[1]{\left( #1 \right)}
\newcommand{\mkArray}[1]{\left[ #1 \right]}
\newcommand{\voidType}{\mathtt{void}}
\newcommand{\boolType}{\mathtt{bool}}
\newcommand{\intType}{\mathtt{int}}
\newcommand{\arrayType}[1]{#1\mathtt{[]}}
\newcommand{\tupleType}[1]{\left(#1\right)}

% Layout
\newcommand{\header}[1]{\addvspace{7pt}\textbf{{#1}}.}

\title{SafetySharp Semantics}
\author{Axel Habermaier, Johannes Leupolz}

\begin{document}

\maketitle

\section{Conventions}
Throughout the remainder of this report, we use the following conventions in all formal definitions whereever appropriate:

\header{Optional Elements and Failure Elements}
	We use $\missing$ and $\failure$ to denote the missing element and the failure element respectively. For some domain $A$, it is
	always true that $\missing \notin A$ and $\failure \notin A$. We define the liftings $\liftMissing{A} := A \cup
	\missing$, $\liftFailure{A} := A \cup \failure$, and $\liftMissingFailure{A} := A \cup \varepsilon \cup	\bot$.
	For some $a \in A$, we know that $a \neq \missing$ and $a \neq \failure$, whereas for $\liftMissing{a} \in
	\liftMissing{A}$, we only know that $\liftMissing{a} \neq \failure$, but $\liftMissing{a}$ might be $\missing$ or
	any element of $A$. When we write $a \in \liftMissing{A}$ or $a = f(a')$ for some function $f : A \rightarrow
	\liftMissing{A}$, $a$ cannot be $\missing$, so the selection or assignment is only possible if the selected or
	returned value is not $\missing$. For example, this allows us to shorten some side conditions in function definitions with case
	distinctions. The following two definitions of function $g : A \rightarrow \liftFailure{A}$ illustrate the usage of this
	convention.
	With function $f$ defined as above, the following two definitions of $g$ are equivalent:
	\begin{align*}
		g(a') = \begin{cases}
			a & \text{if } \quad a = f(a') \\
			\bot & \text{if } \quad \missing = f(a')
		\end{cases}
		&&&
		g(a') = \begin{cases}
			\liftMissing{a} & \text{if } \quad \liftMissing{a} = f(a') \wedge \liftMissing{a} \neq \missing	\\
			\bot & \text{if } \quad \missing = f(a')
		\end{cases}
	\end{align*}

\header{Functions}
	Oftentimes, the codomain of some function includes the failure element. Usually the definitions of those functions contain case distinctions.
	Whenever we do not specify all cases for such a function, the remaining cases are implicitly assumed to return $\failure$.
	Alternatively, we use the word ``otherwise'' to denote a side condition that is true if the side conditions of all other cases are not true. We do not use partial functions.

	It is sometimes necessary to change the value a function returns for a specific input. We establish the notation $f[a \mapsto b]$
	for some function $f : A \rightarrow B$ that serves this purpose and is defined as follows:
	\begin{equation*}
	f[a \mapsto b](a') = \begin{cases}
		b & \text{if } \quad a' = a \\
		f(a') & \text{otherwise}
	\end{cases}
	\end{equation*}

	For convenience, we use the following notations to update several values at the same time: 
	$f[a_1 \; a_2 \mapsto b] = f[a_1 \mapsto b][a_2 \mapsto b]$, $f[a_1 \; a_2 \mapsto b_1 \; b_2] = f[a_1 \mapsto b_1][a_2 \mapsto b_2]$, and $\_$ represents all
	input values, such that $f[ \_ \mapsto b](a)$ returns $b$ for all input values $a$.	This is only well-defined if all updated input values are
	mutually distinct.

\header{Metavariables}
	Words starting with a capital letter denote a semantic domain. For instance, the set of two-dimensional vectors might be defined as 
	$\mathit{Vector2} = \mathbb{R} \times \mathbb{R}$. To denote a single element of a domain, we use metavariables that are identical to the domain's
	name starting with a lower-case letter; in this example, $\mathit{vector2}$. 
	The default metavariable for a domain can be overridden by explicitly specifying the metavariable 
	when defining the domain, such as $v \in \mathit{Vector2} = \mathbb{R} \times \mathbb{R}$. In this case, $v$ denotes an element of domain $\mathit{Vector2}$.
	We also use metavariables in variously decorated forms such as $v'$, $v''$, $v_0$, $v_1$, and so on. 
	We adorn a domain's metavariable with an arrow at the top to denote a list of elements of the domain, for example
	$\toList{v} \in \ToList{\mathit{Vector2}}$.
 
\header{Handling of Tuples}
	Most of the semantic domains we deal with in the following chapters consist of several subdomains. 
	Let $X = A \times B \times C \times D \times \ToList{E}$ be a domain and $x \in X$ be
	an element of that domain. To get the projection of $x$ to one of its subdomains, we subscript $x$ with the metavariable of the
	domain we want to project to. For instance, to get $x$'s projection to domain $D$, we write $x_d$. We use the same convention if the projected subdomain is
	a list, so $x_{\toList{e}}$ projects $x$ to the element in $\ToList{E}$.
	
	We often have a complex tuple type instance like $(a, b, c, d, \toList{e}) \in X$, but in a rule or
	function we actually only need a few of the variables. For example, suppose we only need to use $c$ and $\toList{e}$ in a rule. In that case,
	we write $x \triangleright c, \toList{e}$.
	This way, we can reference the entire tuple with $x$ and do not need any projections to access $x$'s subdomains. When using this projection syntax, we list the
	metavariables of the subdomains in the order they were specified in the definition of $X$. If this notational convention causes any ambiguities, we either
	resort to the standard tuple notation or list all subdomains in our projection notation. We use a notation similar to the
	projection syntax to update only some of a tuple's values: If $x \triangleright a, b, c, d, \toList{e}$, we write $x' = x
	\triangleleft a', \toList{e'}$ as an alternative to $x' = (a', b, c, d, \toList{e'})$.  

\section{Formal Programs}

\subsection{Domains}

Formal programs support integers and Boolean literals as well as variables of those types. A value is either a Boolean or integer
value.
\begin{align*}
    \bool \in \Bool &:= \{ \true, \false \}
    \\
    \integer \in \Integer &:= \{ \ldots, -2, -1, 0, 1, 2, \ldots \}
    \\
    \var \in \Var
    \\
    \val \in \Val &:= \Bool \mid \Integer
\end{align*}

During execution, the \textit{variable environment} stores the values of all variables, provided that a value has previously been
assigned to a variable:
\begin{align*}
	\varEnv \in \VarEnv := \Var \rightarrow \liftMissing{\Val}
\end{align*}

\subsection{Expressions}

Formal programs support only side effect free expressions. Expressions consist of literal values, read operations of variables as
well as a small set of unary and binary operator applications.
\begin{align*}
	\uop \in \Uop & := - \mid \neg
	\\
    \bop \in \Bop & := {} + {} \mid {} - {} \mid {} \ast {} \mid {} / {}
                       \mid {} \% {} \mid {} > {} \mid {} < {} \mid {} \geq {} 
                       \mid {} \leq {} \mid {} = {} \mid {} \neq {}
	\\
	\expr \in \Expr & := 
		\val \mid \var \mid
		\unaryExpr{\uop\;}{\expr} \mid
		\binaryExpr{\;\bop\;}{\expr_1}{\expr_2}
\end{align*}

The function $\ExprSem{-} : \Expr \rightarrow \VarEnv \rightarrow \Val$ defines the semantics of
expressions.
Its definition is omitted for the time being.

\subsection{Statements}

Formal programs support skip statements, variable assignments, sequential compositions and guarded commands.
\begin{align*}
	\stm \in \Stm & :=
		\skipStm \mid 
		\assignVarStm{\var}{\expr} \mid 
		\seqStm{\stm_1}{\stm_2} \mid
		\guardedStm{\guardedClause{\expr}{\stm}}
\end{align*}

The nondeterministic rule system $\stmArrow$ defines the natural semantics of statements.
\begin{align*}
	\mathit{Configurations} : & \;\configuration{\stm, \varEnv} \in \Stm \times \VarEnv, \; \varEnv \in \VarEnv
	\\
	\mathit{Transitions} : & \;\configuration{\stm, \varEnv} \stmArrow \varEnv'
\end{align*}

The skip statement has no effect on the variable environment and is therefore simply discarded.
\begin{equation*}
	\tag{skip}
	\configuration{\skipStm, \varEnv}
		\stmArrow
	\varEnv
\end{equation*}

Assignments to a variable update the variable environment accordingly, replacing the variable's current value with the value
obtained from the evaluation of the assigned expression.
\begin{equation*}
	\tag{assign}
	\configuration{\assignVarStm{\var}{\expr}, \varEnv}
		\stmArrow
	\varEnv[\var \mapsto \ExprSem{\expr} \varEnv]
\end{equation*}

Sequential composition executes the two statements sequentially, passing down the modified variable environment.
\begin{equation*}
	\tag{seq}
	\dfrac
	{
		\configuration{\stm_1, \varEnv}
			\stmArrow
		\varEnv'
		\qquad
		\configuration{\stm_2, \varEnv'}
			\stmArrow
		\varEnv''
	}
	{
		\configuration{\seqStm{\stm_1}{\stm_2}, \varEnv}
			\stmArrow
		\varEnv''
	}
\end{equation*}

Guarded commands nondeterministically choose and execute a statement whose corresponding guard evaluates to true. If no guard
evaluates to true, the rule cannot be applied and hence the program is ``stuck''.

\begin{equation*}
	\tag{cmd}
	\dfrac
	{
		\configuration{\stm_i, \varEnv}
			\stmArrow
		\varEnv'
	}
	{
		\configuration{\guardedClause{\expr_1}{\stm_1} \; \ldots \; \guardedClause{\expr_n}{\stm_n}, \varEnv}
			\stmArrow
		\varEnv'
	}
	\qquad \text{if} \quad \ExprSem{e_i} \varEnv = \true
\end{equation*}

A denotational definition of the statement semantics might be able to better highlight the nondeterministic nature of guarded
commands. Therefore, let the semantic relation $\StmSem{-} : \Stm \rightarrow \ToSet{(\VarEnv \times \VarEnv)}$ define the
denotational semantics of statements as follows:
\begin{align*}
	\StmSem{\skipStm} &= \{ (\varEnv, \varEnv) \mid \varEnv \in \VarEnv \}
	\\
	\StmSem{\assignVarStm{\var}{\expr}} &= \{ (\varEnv, \varEnv[\var \mapsto \ExprSem{\expr} \varEnv]) \mid \varEnv \in \VarEnv \}
	\\
	\StmSem{\seqStm{\stm_1}{\stm_2}} &= \StmSem{\stm_2} \circ \StmSem{\stm_1}
	\\
	\StmSem{\guardedClause{\expr_1}{\stm_1} \ldots \guardedClause{\expr_n}{\stm_n}} &= 
		\bigcup\nolimits_{i = 1}^n 
		{\{ (\varEnv, \varEnv') \mid \ExprSem{e_i} \varEnv = \true \wedge (\varEnv, \varEnv') \in \StmSem{\stm_i} \}}
\end{align*}

Alternatively, we could define the semantic relation $\StmSem{-}$ by reusing rule system $\stmArrow$:
\begin{equation*}
	\StmSem{\stm} := \{ (\varEnv, \varEnv') \in \VarEnv \times \VarEnv \mid \configuration{\stm, \varEnv} \stmArrow \varEnv' \}
\end{equation*}

\subsection{Semantics of Formal Programs}

Formal programs have a set of global and local variables. The program is executed in a series of steps, with each step executing
the exact same sequence of statements. There are global variables that are initialized once and whose values are preserved between
different steps of the program. Additionally, there are step-local variables whose values are not preserved between different
program steps.
\begin{equation*}
	\prog \in \Prog := \ToSet{\Var} \times \ToSet{\Var} \times (\Var \rightarrow \ToSet{\Val}) \times \Stm
\end{equation*}

The first set of variables represents the global variables defined by the program, whereas the second set describes the step-local
ones. The function is used to determine the initial values of all (global) variables; there might be multiple initial
values for each variable. The statement of the program is executed once during each step.

The semantics of a program is a set of \textit{traces}, with each trace being an infinite sequence $\varEnv_0 \,
\varEnv_1 \, \varEnv_2 \ldots$ of variable environments:
\begin{multline*}
	\mathit{traces}(\prog \triangleright \toSet{\var}_g, \toSet{\var}_l, \mathit{init}, \stm) :=
	\\ \shoveright{
	\{ \varEnv_0 \, \varEnv_1 \, \varEnv_2 \ldots \mid 
		\forall \var \in \toSet{\var}_g \colon \varEnv_0(\var) \in \mathit{init}(\var) \qquad \qquad}
	\\ \wedge
		\forall i \in \mathbb{N} \colon \configuration{\stm, \varEnv_i[\toSet{\var}_l \mapsto \missing]} \stmArrow \varEnv_{i+1}
		\}
\end{multline*}

\end{document}
